\documentclass{beamer}

% Russian-specific packages
\usepackage[T2A]{fontenc} % поддержка специальных русских символов
\usepackage[english,russian]{babel}
\usepackage[utf8]{inputenc}

% русские переносы
\usepackage{hyphenat}
% \hyphenation{ма-те-ма-ти-ка вос-ста-нав-ли-вать}

% Стиль презентации
\usetheme{Warsaw}
\usecolortheme{crane}

\begin{document}

\title[Развитие ППРЭ]
{Развитие метода разностной эволюции
для поиска параметров математических моделей}
\author[Свичкарев Анатолий]
{Студент: Свичкарев Анатолий, группа 53601/4\\
Научный руководитель: Козлов Константин Николаевич}
\institute[СПбПУ]
{Санкт-Петербургский Государственный
Политехнический Университет\\Петра Великого}
\date{март, 2016}

% Создание заглавной страницы
\frame{\titlepage} 

\begin{frame}
\frametitle{Введение}
\end{frame}

\begin{frame}
\frametitle{Цель}
\textbf{Улучшить взаимодействие}

\bigskip
Интерпретатор может
быть встроен в программу ППРЭ,
что позволит запускать
нужное число копий один раз,
и, тем самым сократить время вычислений,
для некоторых задач в разы.
\end{frame}

\begin{frame}
\frametitle{Задачи}
\end{frame}

\begin{frame}
\frametitle{Разностная эволюция}
\end{frame}

\begin{frame}
\frametitle{Полностью параллельная разностная эволюция}
\end{frame}

\begin{frame}
\frametitle{Методика экспериментов}
\end{frame}

\begin{frame}
\frametitle{Статистическая обработка}
\end{frame}

\begin{frame}
\frametitle{Выводы}
\end{frame}

\begin{frame}
\frametitle{Используемая литература}
\end{frame}

\begin{frame}
\frametitle{Благодарности}
\end{frame}

\end{document}


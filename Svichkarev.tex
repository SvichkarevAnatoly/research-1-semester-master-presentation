\documentclass{beamer}

% Russian-specific packages
\usepackage[T2A]{fontenc} % поддержка специальных русских символов
\usepackage[english,russian]{babel}
\usepackage[utf8]{inputenc}

% русские переносы
\usepackage{hyphenat}
% \hyphenation{ма-те-ма-ти-ка вос-ста-нав-ли-вать}

% Стиль презентации
\usetheme{Warsaw}
\usecolortheme{crane}

\begin{document}

% Оформление титульного листа
\title[Развитие ППРЭ]
{Развитие метода разностной эволюции
для поиска параметров математических моделей}
\author[Свичкарев Анатолий]
{Студент: \textbf{Свичкарев Анатолий, группа 53601/4}\\
Научный руководитель: \textbf{Козлов Константин Николаевич}}
\institute[СПбПУ]
{Санкт-Петербургский Государственный
Политехнический Университет\\Петра Великого}
\date{март, 2016}

% Создание заглавной страницы
\frame{\titlepage} 

\begin{frame}{Введение}
Математические модели в биоинформатике в
большинстве случаев создаются в таких компьютерных
системах расчетов как \textbf{R, MATLAB, Octave} и др.
\bigskip

Нахождение параметров в таких моделях требует
многократного вычисления решений, что влечет
большие накладные расходы на запуск того или иного
интерпретатора.
\end{frame}

\begin{frame}{Цель}
\textbf{Улучшить взаимодействие}

\bigskip
Интерпретатор может
быть встроен в программу ППРЭ,
что позволит запускать
нужное число копий один раз,
и, тем самым сократить время вычислений,
для некоторых задач в разы.
\end{frame}

\begin{frame}{Задачи}
\end{frame}

\begin{frame}{Разностная эволюция}
\end{frame}

\begin{frame}{Полностью параллельная разностная эволюция}
\end{frame}

\begin{frame}{Методика экспериментов}
\end{frame}

\begin{frame}{Статистическая обработка}
\end{frame}

\begin{frame}{Выводы}
\end{frame}

\begin{frame}{Используемая литература}
\begin{thebibliography}{Dijkstra, 1982}
    \bibitem[Salomaa, 1973]{Salomaa1973}
        A.~Salomaa.
        \newblock {\em Formal Languages}.
        \newblock Academic Press, 1973.
    \bibitem[Dijkstra, 1982]{Dijkstra1982}
        E.~Dijkstra.
        \newblock Smoothsort, an alternative for sorting in situ.
        \newblock {\em Science of Computer Programming}, 1(3):223--233, 1982.
\end{thebibliography}
\end{frame}

\begin{frame}{Благодарности}
\end{frame}

\end{document}

